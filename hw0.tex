% Created 2013-11-27 Wed 21:55
\documentclass[11pt]{article}
\usepackage[utf8]{inputenc}
\usepackage[T1]{fontenc}
\usepackage{fixltx2e}
\usepackage{graphicx}
\usepackage{longtable}
\usepackage{float}
\usepackage{wrapfig}
\usepackage{rotating}
\usepackage[normalem]{ulem}
\usepackage{amsmath}
\usepackage{textcomp}
\usepackage{marvosym}
\usepackage{wasysym}
\usepackage{amssymb}
\usepackage{hyperref}
\tolerance=1000
\author{Tom O'Brien}
\date{\today}
\title{Homework 0}
\hypersetup{
  pdfkeywords={},
  pdfsubject={},
  pdfcreator={Emacs 24.2.93.1 (Org mode 8.2.1)}}
\begin{document}

\maketitle
\section{Obtain a copy of SICP}
\label{sec-1}
SICP is available both in \href{http://www.amazon.co.uk/Structure-Interpretation-Computer-Electrical-Engineering/dp/0262510871/ref%3Dsr_1_1?ie%3DUTF8&qid%3D1385586049&sr%3D8-1&keywords%3Dstructure%2Band%2Binterpretation%2Bof%2Bcomputer%2Bprograms}{physical} and \href{https://mitpress.mit.edu/sicp/full-text/book/book.html}{electronic} forms.  The latter
is freely available and is licensed under a Creative Commons
NonCommercial License.

Download (or order!) a copy of the text, and read \href{https://mitpress.mit.edu/sicp/full-text/book/book-Z-H-9.html#%25_chap_1}{the introduction to Section 1}.

\section{Set up your development environment}
\label{sec-2}
Lisp has many dialects, e.g. Common Lisp, Clojure (Lisp on the JVM), GOOL
(developed at Naughty Dog and \href{http://all-things-andy-gavin.com/2011/03/12/making-crash-bandicoot-gool-part-9/}{used in the development of the Crash Bandicoot series of games})
and Scheme, the dialect used in this course.  Scheme itself comes in many different flavours,
and each flavour has its own quirks and installation process.

The suggested flavour of Scheme used in this course is \href{https://www.gnu.org/software/mit-scheme/}{MIT/GNU Scheme}, though \href{http://racket-lang.org/}{Racket} is also
a popular choice.  Both have their own associated development environments,
and this part of the homework is to get you set-up with one of them.  Alternatively,
both Emacs and Vim support integration with a running MIT/GNU Scheme or Racket process.
\subsection{MIT/GNU Scheme}
\label{sec-2-1}
\begin{itemize}
\item \href{https://www.gnu.org/software/mit-scheme/}{Download} MIT/GNU Scheme for your development platform.
\item Install and open the MIT/GNU Scheme implementation either from
the installed Application or by running
\begin{verbatim}
mit-scheme --edit
\end{verbatim}
at the command line.
\item You are now in Edwin, the MIT/GNU Scheme editor. It is very
similar to Emacs, though written in Scheme.  If this is the first
time in an Emacs-like environment, enter and complete the
interactive tutorial by executing the key-chord \texttt{Ctrl-h t} (Emacs
and Edwin abbreviate this to \texttt{C-h t}.
\end{itemize}

\subsection{Racket}
\label{sec-2-2}
\begin{itemize}
\item \href{http://racket-lang.org/download/}{Download} Racket for your development platform.
\item Open the installed DrRacket application for a Racket development
environment.
\item \textbf{Important}: Racket departs from Scheme in certain ways that make
some of the exercises in SICP impossible.  This can be remedied
by adding
\begin{verbatim}
#lang planet neil/sicp
\end{verbatim}
to the top of any Scheme file you're editing (cf. \href{http://www.neilvandyke.org/racket-sicp/}{Neil Van Dyke}'s
page for more details).  The first time you run a program with
this header, DrRacket will download the required packages.
\end{itemize}
% Emacs 24.2.93.1 (Org mode 8.2.1)
\end{document}