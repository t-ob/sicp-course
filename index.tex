% Created 2013-11-27 Wed 21:56
\documentclass[11pt]{article}
\usepackage[utf8]{inputenc}
\usepackage[T1]{fontenc}
\usepackage{fixltx2e}
\usepackage{graphicx}
\usepackage{longtable}
\usepackage{float}
\usepackage{wrapfig}
\usepackage{rotating}
\usepackage[normalem]{ulem}
\usepackage{amsmath}
\usepackage{textcomp}
\usepackage{marvosym}
\usepackage{wasysym}
\usepackage{amssymb}
\usepackage{hyperref}
\tolerance=1000
\author{Tom O'Brien}
\date{\today}
\title{SICP}
\hypersetup{
  pdfkeywords={},
  pdfsubject={},
  pdfcreator={Emacs 24.2.93.1 (Org mode 8.2.1)}}
\begin{document}

\maketitle
\tableofcontents


\section{Introduction}
\label{sec-1}
\begin{quote}
Lisp is worth learning for the profound enlightenment experience you will have when you finally get it; that experience will make you a better programmer for the rest of your days, even if you never actually use Lisp itself a lot.

-- Eric Raymond, "How to Become a Hacker"
\end{quote}
The Structure and Interpretation of Computer Programs is regarded as
one of the classic textbooks of Computer Science.
% Emacs 24.2.93.1 (Org mode 8.2.1)
\end{document}